% \begingroup
% \let\clearpage\relax
% \chapter{结论与展望}

% 本文介绍了一种基于凸基学习器的元学习方法,用于少样本学习。
% 通过利用对偶形式和KKT条件,可以实现计算和内存高效的元学习,
% 特别适用于少样本学习问题。与最近邻分类器相比,
% 线性分类器在适度增加计算成本的情况下提供更好的泛化能力(如表\ref{table:3}所示)。
% 我们的实验表明,正则化线性模型可以在减少过拟合的同时实现显著更高的嵌入维度。

% 未来的研究方向是探索其他凸基学习器,例如核SVM。这将允许更多的训练数据可用于任务,从而逐步增加模型容量的能力。
% \endgroup