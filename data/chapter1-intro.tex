

\chapter{绪论}

\section{课题研究背景}

从有限的例子中学习是人类智慧的特征,但它仍然是现代机器学习系统的挑战。这个问题引起了机器学习界的极大关注,在机器学习界,
只有少数几个例子的机器学习被认为是一个元学习问题。其目标是在只有少数训练数据的任务中最大限度的减少泛化误差。
典型的方法由一个将输入域映射到特征空间的嵌入模型和一个将特征空间映射到任务变量的基本学习器组成。元学习的目标是
学习一个嵌入模型,从而使基础学习器在不同的任务中具有良好的泛化能力。

\section{课题研究意义}

虽然目前存在许多基础学习器的选择,但最临近分类器以及其变体是最重要的。
由于分类规则简单,该方法在低数据系统中具有良好的扩展性,因此该方法很受欢迎。
然而在低数据情况下,经过鉴别训练的线性分类器的表现往往优于最临近分类器,因为他们能很好的利用反面例子,
通常更丰富,从而更好的学习类别的边界。
此外,他们可以通过适当的正则化,如权值系数性和或范数等,有效的利用高维特征嵌入来控制模型的容量。


\section{课题研究内容}



