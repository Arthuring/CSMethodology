

\chapter{模块1(问题与背景)}

\section{课题研究背景}

人类能够很简单的从几个有限的样例中提取事物的特征并且进行区分,但是这对于现代机器学习来说还是很大的挑战。
在传统的机器学习方法中,模型的泛化能力通常需要大量的数据来进行训练,并且很难在新的任务上进行迁移学习。
在机器学习领域,从数量很少的样例和数据中进行特征提取和分类的问题称为元学习问题,这个问题目前得到了机器学习界的广泛关注
元学习的目标是在只有少数训练数据的情况下,最大程度的减小泛化误差。
经典的元学习模型包括两个部分,分别是将输入域映射到特征空间的嵌入模型,和将特征空间映射到目标变量的基本学习器。
元学习的目标是得到一个嵌入模型,使得基础学习器在不同任务中具有较好的泛化能力。

虽然目前有很多可以选择的基础学习器,但最临近分类器以及其变体是最重要的。
由于分类规则简单,该方法在数据较少的系统中具有很好的扩展性,因此该方法很受欢迎。
然而在数据规模小的情况下,由于鉴别线性分类器能很好的利用更加丰富的反面数据,更好的学习类别的边界,从而使其表现优于最邻近分类器。
此外,他们可以通过适当的正则化,如权值系数性和或范数等,有效的利用高维特征嵌入来控制模型的容量。


\section{课题研究内容}

本文研究了以线性分类器作为基础学习器的元学习问题。本文使用线性支持向量机构建一个分类器,
给定一组已标记的实例用于训练,并在同一任务下的新样本集计算泛化误差。
由于元学习的目标是让模型具有良好泛化性,因此需要在不同任务之间最小化泛化误差,这要求我们以循环优化的方法
来训练线性分类器,这带来了很大的计算量。因此可计算性是此问题的关键。

然而,线性模型的目标函数是通常是凸的,因此这个问题可以被有效解决。
在小样本的环境下,凸优化可以使元学习变得高效。
本文观察到凸的性质中引出的两个额外特性,优化的隐式可微性和分类器的低秩特性\cite{barratt2018differentiability,gould2016differentiating}。
第一个特性允许使用一个已有的凸优化模型估计最优值,并隐式地微分最优性条件或KKT条件来训练嵌入模型。
第二个特性是对于小样本学习,对偶形式中的待优化变量数目远小于特征维数,通过构造对偶优化问题可以大大减少优化变量的个数。


为此,本文将可微的二次规划求解器\cite{amos2017optnet}和不同的线性分类器综合起来。利用以上两个特性,本文实现了在
计算成本略有增加的情况下提供了比最临近分类器更大的收益。
在流行的小规模数据集中,例如 miniImageNet\cite{vinyals2016matching,vilalta2002perspective},tieredImageNet\cite{ren2018meta},CIFAR-FS\cite{bertinetto2018meta} 和 FC100\cite{oreshkin2018tadam},本文方法在5路单次和5次分类中取得了最优性能,

\chapter{相关工作}
元学习探究学习器的哪些方面(通常称为偏差或先验)可以影响其在不同任务上的泛化能力\cite{schmidhuber1987evolutionary,thrun1998lifelong,vilalta2002perspective}。。
用于少量学习的元学习方法可以大致分为三组
一是基于梯度的方法\cite{ravi2017optimization,finn2017model},它通过梯度下降方法寻找和修改嵌入模型的参数。
二是最近邻方法\cite{vinyals2016matching,snell2017prototypical},它在样本的嵌入特征上学习基于距离的预测规则。
例如,原型网络通过样本的中值来嵌入表示每个类别,分类规则基于到最近类别均值的距离\cite{snell2017prototypical}。
再例如匹配网络,它使用训练数据上的嵌入来学习类别密度的核密度估计(该模型也可以被解释为对训练样例的一种注意力形式)\cite{vinyals2016matching}。
三是基于模型的方法\cite{mishra2017simple,munkhdalai2018rapid},学习一个参数化的预测器来估计模型参数,
例如,使用循环网络来预测参数,类似于参数空间中少量步的梯度下降方法。
虽然基于梯度的方法是通用的,但嵌入维度越多,模型更容易过拟合\cite{mishra2017simple,rusu2018meta}。
最近邻方法具有简洁性,并且在少样本环境中扩展性好。但其因为缺少特征选择机制,对噪声的鲁棒性较差。

本文的工作主要与用后向传播进行过程优化的技术相关。
Domke\cite{domke2012generic} 提出了一种基于固定步数的梯度下降和自动微分计算梯度的通用方法。
但是由于需要计算梯度,优化器的优化过程中间值需要被记录,这会需要很大的存储空间,应用于规模较大
的问题是不现实的。
然而,优化器的轨迹(中间值)需要存储以计算梯度,这可能会对大问题造成限制。
Maclaurin 等人考虑了存储开销问题\cite{maclaurin2015gradient},他们研究了深度网络优化轨迹的低精度表示。
如果可以在分析上找到优化的最小值,例如无约束的二次最小化问题,分析的计算梯度也可以被接受。
这个成果已经应用于低层视觉问题中\cite{tappen2007learning,schmidt2014shrinkage}。

本文的方法使用线性分类器,因为它能够规划为凸学习问题。
特别是对于目标函数是一个二次规划问题(QP),可以基于梯度技术高效的获得全局最优解。
此外,凸问题的解可以由它们的Karush-Kuhn-Tucker(KKT)条件所描述,
这使得我们可以使用隐函数定理通过学习者\cite{krantz2002implicit}进行反向传播。
具体而言,本文使用了Amos和Kolter的公式化方法\cite{amos2017optnet},
该方法提供了计算QPs及其梯度的高效GPU程序。虽然他们将这个框架应用于学习约束满足问题的表示,
但由于出现的问题规模通常很小,因此它也非常适合少样本学习。

尽管文中的实验主要关注合页损失与l2正则化的线性分类器,
但这个框架可以与其他损失函数和非线性核一起使用。
例如岭回归学习器可以在本文的框架内实现\cite{bertinetto2018meta},因此这两者在本文可以进行直接比较。