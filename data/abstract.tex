% !Mode:: "TeX:UTF-8"

% 中英文摘要
\begin{cabstract}
    许多用于小样本学习的元学习方法依赖于简单的基础学习器,例如最近邻分类器。
    然而,即使在少样本的情况下,经过区分训练的线性预测器也可以提供更好的泛化。
    本文使用这些预测器作为基础学习器来学习小样本学习的表示,
    并证明了它们在一系列小样本识别基准中在特征大小和性能之间提供更好的权衡。
    本文的目标是学习在新类别的线性分类规则下可以很好地泛化的特征嵌入。
    为了有效地解决目标,本论文利用了线性分类器的两个特性,
    即凸问题最优条件的隐式微分和优化问题的对偶公式,提出了名为MetaOptNet的元学习方法,
    使得能够在适度增加计算开销的情况下使用具有改进泛化的高维嵌入,在miniImageNet、tieredImageNet、
    CIFAR-FS 和 FC100等小样本学习基准数据集上的实验结果表明了该方法的有效性。

\end{cabstract}

\begin{eabstract}
    Many meta-learning approaches for few-shot learning rely on 
    simple base learners such as nearest-neighbor classifiers.
     However, even in the few-shot regime, discriminatively trained linear predictors can offer better generalization. 
     We propose to use these predictors as base learners to learn representations for 
     few-shot learning and show they offer better tradeoffs between feature size and 
     performance across a range of few-shot recognition benchmarks. 
     Our objective is to learn feature embeddings that generalize well under a 
     linear classification rule for novel categories. To efficiently solve the objective, 
     we exploit two properties of linear classifiers: implicit differentiation of the optimality 
     conditions of the convex problem and the dual formulation of the optimization problem. 
     This allows us to use high-dimensional embeddings with improved generalization at a modest increase 
     in computational overhead. Our approach, named MetaOptNet, achieves state-of-the-art performance on 
     miniImageNet, tieredImageNet, CIFAR-FS, and FC100 few-shot learning benchmarks.
\end{eabstract}